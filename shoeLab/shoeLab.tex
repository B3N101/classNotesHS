\documentclass{report}

\input{preamble}
\input{macros}
\input{letterfonts}

\title{\Huge{AP Physics C}\\Lab: Determining the coefficient of static friction between a shoe and a floor and comparing our shoes to one another.}
\author{\huge{\textbf{Ben Feuer}, Joey Casper, and Owen Barry}}
\date{\today}

\begin{document}

\maketitle
\newpage% or \cleardoublepage
% \pdfbookmark[<level>]{<title>}{<dest>}
\pdfbookmark[section]{\contentsname}{toc}
\pagebreak

\section*{Abstract}
In our lab, we found the coefficient of static friction between our shoes and the ground(or in our case a board). We used meter sticks to determine the angle that we positioned the board when the shoes positioned on it first moved by finding the length of the board and the height from the ground to the top of the board at the time that the shoes first moved. We then tested all three types of shoes we had.

\section*{Hypothesis}
We hypothesized that Owen Barry's Brooks would have the heighest coefficient of static friction, Ben Feuer's adidas running shoes would have the second highest coefficient of static friction, and Joey Casper's lifestyle Allbirds shoes would have the lowest coefficient of static fricition, because of the apparent levels of grip on the bottom of the three shoes.

\section*{Data}

\begin{center}
\begin{tabular}{ |c|c|c|c| } 
 \hline
 Shoe & Length (m) & Height (m) & $ \mu_s $ \\
 \hline
 Brooks & 0.615 & 0.5089 & 1.474 \\ 
 \hline
 Adidas & 0.615 & 0.3263 & 0.6259 \\ 
 \hline
 Allbirds & 0.615 & 0.4757 & 1.220 \\ 
 \hline
\end{tabular}
\end{center}

\subsection*{Sample Calculations}

$$ \mu_s = \tan(\theta) $$ 
$$ :\While{}
  \State 
\EndWhile

\section*{Conclusion}




\end{document}
