\documentclass{report}

\input{preamble}
\input{macros}
\input{letterfonts}

\title{\Huge{AP WORLD HISTORY}\\NOTES}
\author{\huge{BEN FEUER}}
\date{}

\begin{document}

\maketitle
\newpage% or \cleardoublepage
% \pdfbookmark[<level>]{<title>}{<dest>}
\pdfbookmark[section]{\contentsname}{toc}
\tableofcontents
\pagebreak

\chapter{SECTION ZERO - FOUNDATIONS}

\section{Governance (and Geographic Orientation)}
\subsection{Forms of governments}
\dfn{Monarchy}{
  Rule by a central leader
}
\dfn{Oligarchy}{
  Rule by a small elite
}
\dfn{Empires}{
  Governmetns created by aquiring new territory by military conqest or diplomatic pressure.
}

\dfn{State building techniques}{
  \begin{itemize}
    \item Writing of law codes
    \item The development of bureaucracies and systems of tax collection
    \item A reliance on elite classes (such as aristocratic nobles or state officials)
    \item The creation of infrastructure fro transport and communications
    \item The mobilization of labor
    \item The use of religion and other sociocultural norms to justify the right to rule.
  \end{itemize}
}

\dfn{Categories of governments}{
  \begin{itemize}
    \item Centralized
    \item Decentralized
  \end{itemize}
}

\chapter{SECTION ONE}

\section{Broad Themes - Governance (1200-1450)}

\subsection{Europe:}
\begin{itemize}
  \item Fuedal monarchies vs. experiments with city state rule
  \item Papal-imperial struggle and the medieval ideal of Christianity
  \item Mongol rule over Russia (Golden Horde)
  \item Centalization in Byzantium(Constantinople) vs. fall to Ottoman conquest (1453)
\end{itemize}

\subsection{Middle East:}
\begin{itemize}
  \item Abbasid Caliphate (Baghdad) peaks in 800s and fragments over 900s 
  \item Dar al-Islam and  "circle of justice"
  \item Sharia law
  \item Mongol Il-Khanate (mid 1200s to mid 1300s)
  \item Ottomon Empire (1299-1922) and conquest of Byzantium (1453)
\end{itemize}

\subsection{Africa:}
\begin{itemize}
  \item Ghana(800-1200)
  \item Mali(Timbuktu, mid 1200s-1600s, Mansa Musa in 1300s)
  \item Hausa Kingdoms
  \item Ethiopia
  \item Great Zimbabwe (1000-1400)
  \item Swahili city-states
\end{itemize}

\subsection{East (and Central) Asia:}
\begin{itemize}
  \item Song dyansty (960-1279)
  \item Mandate of heaven and bureaucracy (civil service exam)
  \item Yuan(mongol) dyansty (1271 - 1368)
  \item Ming dyansty (1368 -1644)
  \item Chagatai (Mongol) khanate in Central Asia (1200s - mid 1600s)
  \item breakdown of Heian  regime (794-1185) in Japan
\end{itemize}

\subsection{South (and Southeast) Asia and Oceania:}
\begin{itemize}
  \item Post-Gupta disunity in India (600-1200)
  \item Incursion of Delhi  Sultanate (1206 - 1526) vs. Resistance of Hindus states
  \item Sinhalese dyansties in Sri Lanka
  \item Khmer Empire (800s - 1400s) and Sukhothai kingdom (1200s - 1400s)
  \item Srivijayan Empire (500s-1100s) and Majapahit (1293-1500)
    City states in Southeast Asia (Malay sultanates, Melaka)
\end{itemize}

\subsection{Americas:}
\begin{itemize}
  \item Mississippian culture(Cahokai, 700-1500)
  \item City states in Mesoamerica (legacy of Maya, 250-900 and Toltecs, 800s-1100s)
  \item Aztecs
  \item Andean city states
  \item Chimu empire
  :q
  :q
  :q
:q
:q



\end{itemize}

\end{document}
