\documentclass{report}

\input{preamble}
\input{macros}
\input{letterfonts}

\title{\Huge{CSE 215}\\\huge{Homework 4}}
\author{\huge{Ben Feuer}}
\date{\today}

\begin{document}

\maketitle

\qs{Problem 1}{
  Prove that there exists a unique prime number of the form $n^2 + 2n - 3$, where n is a positive integer. \\
  Proof: \\
  Let $p = n^2 + 2n - 3$. We can rewrite this as $p = (n + 3)(n - 1)$. Since $p$ is prime, either $n + 3 = 1$ or $n - 1 = 1$. If $n + 3 = 1$, then $n = -2$, which is not a positive integer. Therefore, $n - 1 = 1$, so $n = 2$. Substituting this back into the original equation, we get $p = 2^2 + 2(2) - 3 = 4 + 4 - 3 = 5$. Since $5$ is prime, it is the unique prime number of the form $n^2 + 2n - 3$.
}

\qs{Problem 2}{
  Prove that for all integers $m$ and $n$, $m + n$ and $m - n$ are either both odd or both even. \\
  Proof: \\
  Let $m$ and $n$ be integers. \\
  Case 1: $m$ and $n$ are both odd. \\
  $ m = 2a + 1 $ and $ n = 2b + 1 $ \\
  $ m + n = 2(a+b) + 2 = 2(a+b+1) = 2k $ \\
  $ m - n = 2(a-b) + 0 = 2k $ \\
  Case 2: $m$ and $n$ are both even. \\
  $ m = 2a $ and $ n = 2b $ \\
  $ m + n = 2(a+b) = 2k $ \\
  $ m - n = 2(a-b) = 2k $ \\
  Case 3: $m$ is odd and $n$ is even. \\
  $ m = 2a + 1 $ and $ n = 2b $ \\
  $ m + n = 2(a+b) + 1 = 2k + 1 $ \\
  $ m - n = 2(a-b) + 1 = 2k + 1 $ \\
  Case 4: $m$ is even and $n$ is odd. \\
  $ m = 2a $ and $ n = 2b + 1 $ \\
  $ m + n = 2(a+b) + 1 = 2k + 1 $ \\
  $ m - n = 2(a-b) - 1 = 2k - 1 $ \\
  Therefore, $m + n$ and $m - n$ are either both odd or both even, if both are the same parity then they are both even, and if they are different parities then they are both odd.
}

\qs{Problem 3}{
  Prove that for all integers a, b, and c, if $a|b$ and $a|c$ then $a|(b - c)$. \\
  Proof: \\
  Let $a$, $b$, and $c$ be integers such that $a | b$ and $a | c$. \\
  By definition, $b = ak$ and $c = al$ for some integers $k$ and $l$. \\
  $b - c = ak - al = a(k - l)$. \\
  Since $k$ and $l$ are integers, $k - l$ is also an integer. \\
  Therefore, $a | (b - c)$.
}

\qs{Problem 4}{
  If $n = 4k + 3$, does 8 divide $n^2 - 1$? \\
  Proof: \\
  $ n^2 - 1 = (4k + 3)^2 - 1 = 16k^2 + 24k + 9 - 1 = 16k^2 + 24k + 8 = 8(2k^2 + 3k + 1) $. \\
  Since $n^2 - 1 = 8(a) $, 8 divides $n^2 - 1$. 
}

\qs{Problem 5}{
  Prove that if r is any rational number, then $2r^2 - r + 1$ is rational. \\
  Proof: \\
  Let $r = \frac{a}{b}$ where $a$ and $b$ are integers and $b \neq 0$. \\
  $2r^2 - r + 1 = 2(\frac{a}{b})^2 - \frac{a}{b} + 1 = 2(\frac{a^2}{b^2}) - \frac{a}{b} + 1 = \frac{2a^2}{b^2} - \frac{a}{b} + 1 = \frac{2a^2 - ab + b^2}{b^2}$. \\
  Since $a$, $b$, and $2a^2 - ab + b^2$ are all integers and $b$ and $b^2$ aren't equal to zero, $2r^2 - r + 1$ is rational.
}

\qs{Problem 6}{
  Prove or disprove: For all integers n and m, if $n - m$ is even, then $n^3 - m^3$ is even. \\
  Proof \\
  Case 1: $n$ and $m$ are both even \\
  $ n = 2a $ and $ m = 2b $ for some integers $ a $ and $ b $. \\
  $ n - m = 2a - 2b = 2(a - b) = 2k $ \\ 
  $ n^3 - m^3 = (2a)^3 - (2b)^3 = 8a^3 - 8b^3 = 8(a^3 - b^3) = 2(4a^3 - 4b^3) = 2k$ \\
  Case 2: $n$ and $m$ are both odd \\
  $ n = 2a + 1 $ and $ m = 2b + 1 $ for some integers $ a $ and $ b $. \\
  $ n - m = 2a + 1 - 2b - 1 = 2(a - b) = 2k $ \\
  $ n^3 - m^3 = (2a + 1)^3 - (2b + 1)^3 = 8a^3 + 12a^2 + 6a + 1 - 8b^3 - 12b^2 - 6b - 1 = 8(a^3 - b^3) + 12(a^2 - b^2) + 6(a - b) = 8(a^3 - b^3) + 12(a + b)(a - b) + 6(a - b) = 8(a^3 - b^3) + 12k + 6k = 2(4(a^3 - b^3) + 6k) = 2k$ \\
  Therefore, $n - m$ being even implies that $n^3 - m^3$ is even.
}

\qs{Problem 7}{
  Prove that the sum of any two odd integers is even. \\
  Proof: \\
  Let $n$ and $m$ be odd integers. \\
  $ n = 2a + 1 $ and $ m = 2b + 1 $ for some integers $ a $ and $ b $. \\
  $ n + m = 2a + 2b + 2 = 2(a + b + 1) = 2k $ \\
  Therefore, the sum of any two odd integers is even.
}

\qs{Problem 8}{
  Prove whether the following statement is valid: For all real numbers $a$ and $b$, if $a < b$ then $a^2< b^2$. \\
  Proof (by counterexample): \\
  Let $a = -2$ and $b = 1$. \\
  $a < b$ since $-2 < 1$. \\
  $a^2 = (-2)^2 = 4$ and $b^2 = 1^2 = 1$. \\
  $a^2 > b^2$ since $4 > 1$. \\
  Therefore, the statement is not valid.
}

\qs{Problem 9}{
  If r and s are both positive integers, is $r^2 + 2rs + s^2$ composite? \\
  Proof: \\
  $r^2 + 2rs + s^2 = (r + s)^2$. \\
  Since $r$ and $s$ are both positive integers, $r + s$ is also a positive integer, that must be greater than 1. \\
  Therefore, $r^2 + 2rs + s^2$ is composite.
}




\end{document}
