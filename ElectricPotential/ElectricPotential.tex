\documentclass{report}

\input{preamble}
\input{macros}
\input{letterfonts}

\title{\Huge{AP Physics I}\\Electric Potential and Electric Potential Energy}
\author{\huge{Ben Feuer}}
\date{}

\begin{document}

\maketitle
\newpage% or \cleardoublepage
% \pdfbookmark[<level>]{<title>}{<dest>}
\pdfbookmark[section]{\contentsname}{toc}
\tableofcontents
\pagebreak

\chapter{}

\section{Equations for Electric potential and electric potential energy} 

\begin{itemize}
  \item $ \textit{V} = \frac{\textit{U}_{elec}}{\textit{q}} $
  \item $ K_f + U_f = K_i + U_i $
  \item $ \Delta K = -q \Delta \textit{V} $
  \item $ \mathrm{U}_{elec} = \frac{kq_1q_2}{r} $
  \item $ \textit{V} = k \frac{q}{r} $
  \item $ V = \sum_{i} \frac{kq_i}{r_i} $
  \item $ V = \frac{x}{d} \Delta \textit{V}_c$
\end{itemize}

\section{Electric potential Energy}

\dfn{Electric Potential Energy}{
  \textbf{Electric potential energy is $\Delta \mathrm{U}_{elec}$}
  \\
  $\mathrm{U}_{elec} = qV $
}


\section{Electric potential}

\dfn{Electric Potential}{
  Ratio of electric potential energy to charge.
}

\dfn{Voltage}{
  Voltage is a measure of electric potential.
  \begin{itemize}
    \item symbol Voltage = \textit{V}
    \item unit Volt = V = $ \frac{\mathrm{J}}{\mathrm{C}} $
  \end{itemize}
}

\newpage

\section{Conservation of Energy for a charged particle moving in an electric Potential \textit{V}}
$ \Delta K = -q \Delta \textit{V} $
\\
$ K_f + q\textit{V}_f = K_i + q\textit{V}_i $
\\
$ \frac{1}{2} mv_f^2 = \frac{1}{2}mv_i^2 + (q\textit{V}_i - q\textit{V}_f) $
\\
$ v_f^2 = v_i^2 + \frac{2}{m}(q\textit{V}_i - q\textit{V}_f) = v_i^2 = \frac{2q}{m} (\textit{V}_i - \textit{V}_f) $

\section{Electric potential in a parallel-plate capacitor}
Parallel Plate Capactitor giving you:

\begin{itemize}
  \item Charges Q plus/minus
  \item $ E = \frac{Q}{\epsilon_0 A} $
  \item Plate seperation - d 
  \item Displacement - x 
\end{itemize}


Equations:
\begin{itemize}
\item $ W = \mathrm{force} \mathrm{x} \mathrm{displacement} = F_{hand}x = qEx $
  \item $ U_{elec} = W = qEx $
\end{itemize}

\section{Equipotential lines}
\dfn{Equipotential lines}{
  Equipotential lines are lines that show equal voltage constantly. 
  \\
  Electric field lines point in the direction of decreasing potential. 
  \\
  Electric field strength in terms of the potential difference between two $ \Delta \textit{V} $ Between two equipotential surfaces a distance d apart. $$ E = \Delta \frac{\textit{V}}{d} $$
}

\section{Capacitance and capacitors}

\dfn{Capacitor}{
  Two conductors with equal but opposite charge. The two conductors are called its electrodes, or plates.
  \\
  The potential difference between the electrodes is directly proportional to their charge.
  \\
  The charge of a capacitor is directly proportional to the potential difference between its electrodes.
  $$ Q = C \Delta \textit{V}_C$$ 
Charge on a capacitor with a potential difference $ \Delta \textit{V}_C $
}

\dfn{Capacitance}{
  The constant of proportionality C between Q and $ \Delta \textit{V}_C $ is called the capacitance of the capacitor.
  \\
  The SI unit for capacitance is called the farad 1 farad $ = 1F = 1 \frac{C}{V} $
}




\end{document}
