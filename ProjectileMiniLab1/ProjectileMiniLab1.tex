\documentclass{report}

\input{preamble}
\input{macros}
\input{letterfonts}

\title{\Huge{Projectile Mini Lab 1}\\\huge{AP Physics C}}
\author{\huge{Ben Feuer}}
\date{Due September 22, 2023}

\begin{document}

\maketitle
\newpage% or \cleardoublepage
% \pdfbookmark[<level>]{<title>}{<dest>}
\pdfbookmark[section]{\contentsname}{toc}
% \tableofcontents
\pagebreak

\section{Introduction}

\textbf{The goal} of this lab is to determine where a marble will land given a 45 degree angle based on data from when the marble was launched at a 0 degree angle.
\\
\textbf{The materials} that are used in this lab include measuring sticks, kinematic equations, a marble, and a marble launcher. 

\section{Procedure}

\textbf{The procedure} for this lab is as follows:
\begin{enumerate}
  \item First, we found the height of the launcher from the ground, and the displacement of the marble when being launched from 0 degrees. 
  \item We then perdicted where the marble would land if it was launched at a 45 degree angle.
  \item Then, we launched the marble at a 45 degree angle, and measured the displacement of the marble from the launcher.
  \end{enumerate}


\section{Diagram of the Procedure}
        
%diagram png goes here 

\begin{figure}[h!]
  \begin{center}
    \includegraphics[width=0.5\textwidth]{diagram.png}
  \end{center}
\end{figure}

\section{Data}

\textbf{The data} that we collected for the height of the launcher, and the displacement of the marble when being launched from 0 degrees is as follows:

\begin{center}
\begin{tabular}{ |c|c|c| } 
 \hline
 Trial & Height of Launcher (m) & Displacement of Marble (m) \\ 
 \hline
 1 & 1.174 & 2.63 \\ 
 2 & 1.174 & 2.60 \\ 
 Avg. & 1.174 & 2.615 \\ 
 \hline

\end{tabular}

\end{center} 

\section{Perdicting the Displacement of the Marble at 45 Degrees}

To perdict the displacement of the marble at 45 degrees, we first found the velocity that the marble at 0 degrees was launched at. 
$$ t = \sqrt{\frac{2h}{g}} = \sqrt{\frac{2(1.174 m)}{9.8 \frac{m}{s^2}}} = 0.49 s $$
$$ v = \frac{\Delta x}{t} = \frac{2.615 m}{0.49 s} = 5.34 \frac{m}{s} $$

We then used the velocity to find the displacement of the marble at 45 degrees by first finding the x and y components of the velocity at 45 degrees to find the time that the marble is in the air to then find the marble's horizontal displacement. 

$$ v_{ix} = v_{iy} = v_i \cos{\theta} = 5.34 \frac{m}{s} \cos{45} = 3.77 \frac{m}{s} $$ 

$$ \Delta y = v_{iy} t + \frac{1}{2} a_y t^2 \implies 3.77 \frac{m}{s} t + \frac{1}{2} (-9.8 \frac{m}{s^2}) t^2 + 1.174m = 0 $$ 

% Quadratic Formula to find t % 

$$ t = \frac{-3.77 \frac{m}{s} \pm \sqrt{(3.77 \frac{m}{s})^2 - 4(0.5)(-9.8 \frac{m}{s^2})(1.174m)}}{2(0.5)(-9.8 \frac{m}{s^2})} = 1.00725s $$
$$ \Delta x_{\mathrm{perdicted}} = v_{ix} t_{45} = 3.77 \frac{m}{s} (1.00725 s) = 3.80 m $$ 

\section{Experimental result}

\textbf{The experimental result} of the displacement of the marble at 45 degrees is as follows:

$$ \Delta x_{45} = 4.08 m $$

\section{Conclusion}

\subsection{Percent Error}

$$ \% Error = |\frac{\Delta x_{45} - \Delta x_{\mathrm{perdicted}}}{\Delta x_{\mathrm{perdicted}}}| \times 100\% = |\frac{4.08m - 3.80m}{3.80m}| \times 100\% = 7.37\% $$

\subsection{Sources of Error}

\textbf{The sources of error} in this lab include the marble not being launched at exactly 45 degrees, and the marble not being launched at the same velocity as the marble was launched at 0 degrees. If the marble was launched at an angle different from 45 degrees, it would be at a lower angle because the marble is launched above the ground would have a higher horizontal velocity and thus travel for a longer distance. And, if the marble wasn't launched at the same velocity as it was launched at 0 degrees, it would be launched a higher velocity. Lastly, factors such as spin and drag on the marble which we didn't account for could also lead to the error; however, those sources would likely slow the marble down.
\end{document}
