\documentclass{report}

\input{preamble}
\input{macros}
\input{letterfonts}

\title{\Huge{CSE 215 Homework 8}}
\author{\huge{Ben Feuer}}
\date{\today}

\begin{document}

\maketitle

\qs{Problem 1}{
  Let A be the set of all strings of length 6 consisting of x's and y's. Then A is denoted $\sum^6$ where $\sum = \{x,y\}$. Define a binary relation R from A to A as follows:
  \\
  For all strings s and t in A, $s R t \iff$ the first four characters of s equal the first four characters of t.
  \\ \\
  a) Is xxyxyx R xxxyxy ? \\
  Answer: No, because the first four characters of xxyxyx are xxyx, and the first four characters of xxxyxy are xxxy, which are not equal. \\
  \\
  b) Is yxyyyx R yxyyxy? \\
  Answer: Yes, because the first four characters of yxyyyx are yxyy, and the first four characters of yxyyxy are yxyy, which are equal. \\
  \\
  c) Is xyxxxx R yxxxxx? \\
  Answer: No, because the first four characters of xyxxxx are xyxx, and the first four characters of yxxxxx are yxxx, which are not equal.
}

\qs{Problem 2}{
  Let A = { 2, 3, 4 } and B = { 6, 8, 10 } and define a binary relation R from A to B as follows: \\
  for all $(x, y) \in A \times B, (x, y) \in R \iff x | y$.
  \\ \\
  a) Is 4 R 6? \\
  Answer: No, because 4 does not divide 6. \\ \\
  b) Is 4 R 8? \\
  Answer: Yes, because 4 divides 8, and 4 is in A and 8 is in B. \\ \\
  c) Is $(3, 8) \in R$? Is $(2, 10) \in R$? \\
  Answer: No and Yes, respectively. 3 does not divide 8, but 2 divides 10. \\ \\
  d) Write R as a set of ordered pairs. \\
  Answer: $R = \{(2, 6), (2, 8), (2, 10), (3, 6), (4, 8)\}$ \\
  Equal to the set of ordered pairs where the first element divides the second element.
}

\qs{Problem 3}{
  Prove that for all integers m and n, m - n is even if and only if both m and n are even or both m and n are odd. \\

  $$  m - n = 2k \iff (m = 2i \land n = 2j) \lor (m = 2i + 1 \land n = 2j + 1) $$
  
  \begin{align*}
    \equiv [(m-n) = 2k \implies (m = 2i \land n = 2j) \lor (m = 2i + 1 \land n = 2j + 1)] \\
    \land [(m=2i \land n = 2j) \lor (m = 2i + 1 \lor n = 2j+1) \implies (m-n) = 2k]
  \end{align*}

  Proof: \\
  Case 1: m and n are both even. \\
  Let $m = 2i$ and $n = 2j$. \\
  Then $m - n = 2i - 2j = 2(i - j) = 2k$ where $k = i - j$. \\
  Thus, if m and n are both even, then m - n is even. \\

  Case 2: m and n are both odd. \\
  Let $m = 2i + 1$ and $n = 2j + 1$. \\
  Then $m - n = 2i + 1 - 2j - 1 = 2(i - j) = 2k$ where $k = i - j$. \\
  Thus, if m and n are both odd, then m - n is even. \\

  Case 3: m is even and n is odd. \\
  Let $m = 2i$ and $n = 2j + 1$. \\
  Then $m - n = 2i - 2j - 1 = 2(i - j) - 1 = 2k - 1$ where $k = i - j$. \\
  Thus, if m is even and n is odd, then m - n is odd. \\

  Case 4: m is odd and n is even. \\
  Let $m = 2i + 1$ and $n = 2j$. \\
  Then $m - n = 2i + 1 - 2j = 2(i - j) + 1 = 2k + 1$ where $k = i - j$. \\
  Thus, if m is odd and n is even, then m - n is odd. \\

  Therefore, m - n is even if and only if both m and n are even or both m and n are odd.
}

\qs{Problem 4}{
  Declare a binary relation S from R to R as: \\
  For all $(x, y) \in R \times R$, $x S y \iff x \ge y$. \\
  Draw the graph of S in the Cartesian plane. \\
  Graph below where the area below the line $y = x$ is shaded in red, including the line itself.
}

\begin{figure}[b!]
  \begin{center}
    \includegraphics[width=0.5\textwidth]{figures/p4Graph.png}
  \end{center}
  \caption{Graph for problem 4}\label{fig:p4Graph}
\end{figure}

\newpage


\qs{Problem 5}{
  Define a binary relation T from R to R as follows: \\
  For all $(x, y) \in R \times R, x T y \iff y = x^2$. \\
  Draw the graph of T in the Cartesian plane. \\
  Answer: The graph of T is the line of the parabola $y = x^2$ which is shown below.
}

\begin{figure}[h!]
  \begin{center}
    \includegraphics[width=0.5\textwidth]{figures/p5Graph.png}
  \end{center}
  \caption{Graph for problem 5}\label{fig:p5Graph}
\end{figure}


\qs{Problem 6}{
  For the following relations, 1) draw the directed graph, 2) determine whether it is reflexive, symmetric, and transitive. Give a counterexample in each case in which the relation does not satisfy one of these properties.

  \textbf{All graphs are shown below.} \\
  a) $R_{1} = \{(0, 0), (0, 1), (0, 3), (1, 1), (1, 0), (2, 3), (3, 3)\} $ \\
  Answer: not reflexive, not symmetric, not transitive. \\
  Reflexivity: (2, 2) is not in the relation. \\
  Symmetry: (0, 3) is in the relation, but (3, 0) is not. \\
  Transitivity: (1,3) is missing but (1, 0) and (0, 3) are present. \\ \\
  b) $R_{2} = \{(2, 3), (3, 2)\} $ \\
  Answer: not reflexive, symmetric, transitive. \\
  Reflexivity: (2, 2) and (3, 3) are not in the relation. \\
  Symmetry: No counterexample. \\
  Transitivity: Vacuously transitive. \\ \\
  c) $R_{3} = \{(0, 1), (0, 2)\} $ \\
  Answer: not reflexive, not symmetric, transitive. \\
  Reflexivity: (0, 0) is not in the relation. \\
  Symmetry: (0, 1) is in the relation, but (1, 0) is not. \\
  Transitivity: Vacuously transitive.
}
\newpage
\begin{figure}[h!]
  \begin{center}
    \includegraphics[width=0.4\textwidth]{figures/p6GraphA.png}
  \end{center}
  \caption{Graph for problem 6 A}\label{fig:p6GraphA}
\end{figure}

\begin{figure}[h!]
  \begin{center}
    \includegraphics[width=0.4\textwidth]{figures/p6GraphB.png}
  \end{center}
  \caption{Graph for problem 6 B}\label{fig:p6GraphB}
\end{figure}

\newpage
\begin{figure}[h!]
  \begin{center}
    \includegraphics[width=0.4\textwidth]{figures/p6GraphC.png}
  \end{center}
  \caption{Graph for problem 6 C}\label{fig:p6GraphC}
\end{figure}


\qs{Problem 7}{
  D is the binary relation defined on R as follows: \\
  For all $ x, y \in R, x D y \iff xy \ge 0 $. \\ \\
  a) Draw a Cartesian Graph of the relation. \\
  Answer: The graph is shown below. \\
  b) Is it Reflexive? \\
  Answer: Yes, because all real numbers multiplied by themselves are greater than or equal to 0, meaning that all real numbers are related to themselves. \\
  c) Is it Symmetric? \\
  Answer: Yes, because if $ xy \ge 0 $, then $ yx \ge 0 $, so the relation is symmetric. \\
  d) Is it Transitive? \\
  Answer: Yes, because if $ xy \ge 0 $ and $ yz \ge 0 $, then $ xz \ge 0 $, so the relation is transitive as x and z must have the same sign, meaning that their product is greater than or equal to 0.
}

\begin{figure}[h!]
  \begin{center}
    \includegraphics[width=0.25\textwidth]{figures/p7Graph.png}
  \end{center}
  \caption{Graph for problem 7}\label{fig:p7Graph}
\end{figure}

\end{document}
