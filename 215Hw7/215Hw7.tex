\documentclass{report}

\input{preamble}
\input{macros}
\input{letterfonts}

\title{\Huge{CSE 215}\\Homework 7}
\author{\huge{Ben Feuer}}
\date{\today}

\begin{document}

\maketitle
\newpage% or \cleardoublepage

\qs{Question 1}{
  Define $f: R -> R$ and $g: R -> R$ by the following formulas: $f(x) = |x|$ for all $x \in R, g(x) = (x^2)^(1/2)$ for all $x \in R$. \\
  Does $f == g$? \\
  Answer: Yes, $f == g$ because $|x| = (x^2)^(1/2)$ for all $x \in R$. \\
  $|x| = (x^2)^(1/2)$ for all $x \in R$ because the absolute value of x is the square root of x squared as the square root of x squared is always positive. \\
}

\qs{Question 2}{
  How many functions are there from a set with three elements to a set with four elements?
  Answer: $4^3 = 64$ \\
  There are 4 choices for each of the 3 elements in the domain, so there are $4^3 = 64$ functions, and there are no restrictions on the function(i.e. injective, surjective).
}

\qs{Question 3}{
  Define functions f and g from R to R by the formulas: for all $x \in R$, 
  $$f(x) = 2x$$
  $$g(x) = (2x^3 + 2x)/(x^2 + 1)$$
  Show that $f == g$.
  $$ g(x) = (2x^3 + 2x)/(x^2 + 1) = 2x(x^2 + 1)/(x^2 + 1) = 2x = f(x)$$
}

\qs{Question 4}{
  For $f(x) = (x+1)/x$ for all real numbers where $x \neq 0$, \\
  a) Is this function onto? or b) Is this function one-to-one? \\
  Answer: This function is one-to-one. 
  Suppose $f(x_{1}) = f(x_{2})$ for $x_{1}, x_{2} \in R$.
  $$ \frac{x_{1}+1}{x_{1}} = \frac{x_{2}+1}{x_{2}} $$
  $$ x_{1}x_{2} + x_{1} = x_{1}x_{2} + x_{2} \text{ cross multiplication}$$ 
  $$ x_{1} = x_{2} \text{ subtraction, yielding equality}$$ 
  Since $f(x_{1}) = f(x_{2})$ implies $x_{1} = x_{2}$, the function is one-to-one.
}
\qs{Question 5}{
  For $f(x) = x/(x^2 + 1)$ for all real numbers x \\
  a) Is this function onto? or b) Is this function one-to-one? \\
  Answer: This function is one-to-one. \\
  Suppose $f(x_{1}) = f(x_{2})$ for $x_{1}, x_{2} \in R$.
  $$ \frac{x_{1}}{x_{1}^2 + 1} = \frac{x_{2}}{x_{2}^2 + 1} $$ 
  $$ x_{1}x_{2}^2 + x_{1} = x_{1}^2x_{2} + x_{2} \text{ cross multiplication}$$ 
  $$ x_{1} = x_{2} \text{ subtraction, yielding equality}$$ 
  Since $f(x_{1}) = f(x_{2})$ implies $x_{1} = x_{2}$, the function is one-to-one.
}

\qs{Question 6}{
  How many one-to-one functions are there from a set with three elements to a set with four elements? \\
  Answer: 24 \\
  There are 4 choices for the first element in the codomain, 3 choices for the second element in the codomain, and 2 choices for the third element in the codomain. \\
  $4 * 3 * 2 = 24$ one-to-one functions.
}

\qs{Question 7}{
  How many onto functions are there from a set with three elements to a set with two elements? \\
  Answer: 6 \\
  $1: x_{1} \to y_{1}, x_{2} \to y_{1}, x_{3} \to y_{2}$ \\
  $2: x_{1} \to y_{1}, x_{2} \to y_{2}, x_{3} \to y_{1}$ \\
  $3: x_{1} \to y_{1}, x_{2} \to y_{2}, x_{3} \to y_{2}$ \\
  $4: x_{1} \to y_{2}, x_{2} \to y_{1}, x_{3} \to y_{1}$ \\
  $5: x_{1} \to y_{2}, x_{2} \to y_{1}, x_{3} \to y_{2}$ \\
  $6: x_{1} \to y_{2}, x_{2} \to y_{2}, x_{3} \to y_{1}$ \\
  This is also equivalent to the number of functions from a set with three elements to a set with two elements, which is $2^3 = 8$, minus the number of functions that are not onto, which is $2 * 1 = 2$ when all $y_1$ is mapped to all elements of X and $y_{2}$ is mapped to all elements of X. \\
}

\qs{Question 8}{
  In a group of 30 people, at least how many must have been born in the same month? \\
  Answer: 3 \\
  There are 12 months in a year, so if there are 30 people, there must be at least 3 people born in the same month by the Pigeonhole Principle as the number of people per month codomain is 3 for 6 months and 2 for the other 6 months. From 25 people onwards, there must be at least 3 people born in the same month by the Pigeonhole Principle for one of the months.
}

\end{document}
