\documentclass{report}

\input{preamble}
\input{macros}
\input{letterfonts}

\title{\Huge{CSE 215}\\Homework 5}
\author{\huge{Ben Feuer}}
\date{\today}

\begin{document}

\maketitle

\qs{Problem 1}{
Write the first four terms of the following sequence: \\
$e_n = \left\lfloor \frac{n}{2} \right\rfloor * 2$, for all integers $ n\ge 0$. \\
Answer: $0, 0, 2, 2$
}

\qs{Problem 2}{
  Write the first ten terms of the following sequence: \\
$ g_n = \left\lfloor \log_2 n \right\rfloor $ for all integers $ n \ge 1$. \\
Answer: $0, 1, 1, 2, 2, 2, 2, 3, 3, 3$
}

\qs{Problem 3}{
  Find an explicit formula to represent a sequence with the following initial terms: -1, 1, -1, 1, -1, 1. \\
Answer: $a_n = (-1)^n$ where $a_1 = -1$
}

\qs{Problem 4}{
  Find an explicit formula to represent a sequence with the following initial terms: 0, 1, -2, 3, -4, 5. \\
  Answer: $a_n = (-1)^n(n-1)$ where $a_1 = 0$
}

\qs{Problem 5}{
  Find an explicit formula to represent a sequence with the following initial terms: $\frac{1}{2}, \frac{2}{3}, \frac{3}{4}, \frac{4}{5}, \frac{5}{6}, \frac{6}{7}$. \\
  Answer: $a_n = \frac{n}{n+1}$ where $a_1 = \frac{1}{2}$
}

\qs{Problem 6}{
  Compute the summation: $sum_{k=1}^{5} (k+1)$. \\
  Answer: $2 + 3 + 4 + 5 + 6 = 20$
}

\qs{Problem 7}{
  Compute the product: $ \Pi_{k=2}^{4} (k^2)$. \\
  Answer: $4 \cdot 9 \cdot 16 = 36 \cdot 16 = 360 + 216 = 576 $
}

\qs{Problem 8}{
  What is: $ \frac{100!}{98!}$? \\
  Answe: $100 \cdot 99 = 9900$
}

\qs{Problem 9}{
  Reduce the following so that it has no factorial: $\frac{(n-1)!}{(n+1)!}$. \\
  Answer: $\frac{(n-1)!}{(n+1)!} = \frac{(n-1)!}{(n+1)(n)(n-1)!} = \frac{1}{n(n+1)}$
}
\qs{Problem 10}{
  Use mathematical induction to prove that shows the following:
  $ 1^2 + 2^2 + \dots + n^2 \equiv \frac{n(n+1)(2n+1)}{6} $ for all integers $n \ge 1$.\\ \\
  Base case: $n = 1$ \\
  $$ n^2 = 1^2 = \frac{1(1+1)(2(1)+1)}{6} = \frac{1(2)(3)}{6} \equiv 1$$ 

  Inductive Hypothesis: Assume that the formula holds for $n = k$. \\
  Inductive Step: Show that the formula holds for $n = k+1$. \\
  $$ 1^2 + 2^2 + \dots + k^2 + (k+1)^2 = \frac{k(k+1)(2k+1)}{6} + (k+1)^2 $$ $$ \frac{k(k+1)(2k+1) + 6(k+1)^2}{6} = \frac{(k+1)(k(2k+1) + 6(k+1))}{6} $$ $$ \frac{(k+1)(2k^2 + k + 6k + 6)}{6} = \frac{(k+1)(2k^2 + 7k + 6)}{6} = \frac{(k+1)(k+2)(2k+3)}{6}$$
}

\qs{Problem 11}{
  Use mathematical induction to prove that shows the following: $1 + 3 + 5 + \dots + (2n-1) \equiv n^2$ for all integers $ n\ge 1$.\\ \\
Answer: \\
Base case: $n = 1$ \\
$$ 1 = 1^2 \equiv 1 $$ 
Inductive Hypothesis: Assume that the formula holds for $n = k$. \\
Inductive Step: Show that the formula holds for $n = k+1$. \\
$$ 1 + 3 + 5 + \dots + (2k-1) + (2(k+1)-1) = k^2 + 2k + 1 $$ 
$$ k^2 + 2k + 1 = (k+1)(k+1) = (k+1)^2 $$
}

\qs{Problem 12}{
  Use mathematical induction to prove that: $2 + 4 + 6 + ... + 2n \equiv n^2 + n$ for all integers $n \ge 1$ .\\ \\
Answer: \\
Base case: $n = 1$ \\
$$ 2(1) = 1^2 + 1 \equiv 2 $$ 
Inductive Hypothesis: Assume that the formula holds for $n = k$. \\
Inductive Step: Show that the formula holds for $n = k+1$. \\
$$ 2 + 4 + 6 + \dots + 2k + 2(k+1) = k^2 + k + 2(k+1) $$ 
$$ k^2 + k + 2k + 2 = k^2 + 3k + 2 = (k^2 + 2k + 1) + 2k + 1 = (k+1)^2 + (k+1) $$
}

\qs{Problem 13}{
  Use mathematical induction to prove that: $ 4^n - 1 $ is divisible by 3 for each integer $n \ge 1$.\\ \\
Answer: \\

Base case: $n = 1$ \\
$$ (4^1 - 1) = 3 \text{ which is divisible by 3} $$
Inductive Hypothesis: Assume that the formula holds for $n = k$. \\
Inductive Step: Show that the formula holds for $n = k+1$. \\
$$ 4^{k+1} - 1 = 4 \cdot 4^k - 1 = 4 \cdot 4^k - 4 + 3 = 4(4^k - 1) + 3 $$ 
Since $4^k - 1$ is divisible by 3, $4(4^k - 1)$ is also divisible by 3. Therefore, $4^{k+1} - 1$ is divisible by 3.
}

\qs{Problem 14}{
  Use mathematical induction to prove that: $ n^3 - n  $ is divisible by 6 for all integers $n \ge 2$.\\ \\
Answer: \\
Base case: $n = 2$ \\
$$ n^3 - n = 2^3 - 2 = 8 - 2 = 6 \text{ which is divisible by 6} $$ 
Inductive Hypothesis: Assume that the formula holds for $n = k$. \\
Inductive Step: Show that the formula holds for $n = k+1$. \\
$$ (k+1)^3 - (k+1) = (k+1)(k^2 + 2k + 1) - (k+1) $$ 
$$ (k+1)(k^2 + 2k + 1 - 1) = (k+1)(k^2 + 2k) = (k+1)k(k+2) $$
$$ (k+1)(k)(k+2) = (k^3 + 3k^2 + 2k) = (k^3 - k) + 3k^2 + 3k $$ 
$$ = (k^3 - k) + 3k(k+1) $$ 
Since $k^3 - k$ is divisible by 6, and $3k(k+1)$ is divisible by 6, $(k+1)^3 - (k+1)$ is divisible by 6.
}





  
\end{document}
